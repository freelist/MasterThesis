\chapter{Conclusions}\label{ch:conclusions}
In this work we have presented a large view about perception in industrial settings, especially focusing on robotic and machine vision. We analyzed state-of-the-art technologies, theories and their software and hardware implementations particularly looking at 3D object detection and localization and 3D scene reconstruction.

In this settings we have proposed novel benchmarking tools, such as the RAW Dataset, that wants to achieve higher levels of robustness, completeness and consistency in industrially oriented benchmarks, while maintaining at the same time high levels of complexity in order to be as much as possible reliable with complex and accurate software algorithms and tools used nowadays in industrial scenarios. The RAW Dataset is going to be published as one of the few complete and powerful test bed for the industrial community.

Nevertheless, this is still an on going work and many other plans are near to be accomplished before actually achieve the level of completeness needed in such strict and controlled settings. As part of the FlexSight project, the RAW Dataset will be used as first test bed solution for the largely introduced FlexSight Sensor, and given the nature of the project, the RAW Dataset will be augmented for supporting also test cases for other industrial applications, such as deformable object detection and localization.

This work is also the introduction to other already mentioned future works, namely:

\begin{itemize}
	\item Reach a complete coverage of the ground truth for all the scenes of the RAW Dataset;
	\item Deeply analyze the effectiveness of our implementation of the 3D scene reconstruction method;
	\item Build and development of the final prototype of the FlexSight Sensor;
	\item Test the final FlexSight Sensor prototype on the RAW Dataset using all the tools developed in this work;
	\item Development and implementation of novel Deep Learning approaches to the problem of object localization, in terms of 6D pose detection, rotation and translation. A totally neural network approach still does not exist for this kind of problem.
\end{itemize}

\newpage

\begin{acknowledgements}
There are many people that I would like to thank and, therefore, this Chapter is, inevitably, in Italian, and let me leave such a formal language from now on, those words are really coming out of my heart.

In primo luogo, vorrei dedicare questo lavoro alla mia famiglia, i miei genitori e mia sorella. Se nella vita sono diventato la persona che tutti oggi conoscono, devo questo a loro, al loro impegno continuo nello starmi vicino, supportarmi e sostenermi in ogni scelta o difficolt\'a. Un semplice grazie non basta per dimostrare la mia gratitudine nei loro confronti, ma se ci sono persone che meritano queste semplici ma sincere parole, sono loro.

Ringrazio \emph{Linda}, la figura pi\'u importante della mia vita, la mia ragazza, il mio pi\'u importante pensiero. Avere vicino persone che danno la vita per te, che danno tutto per farti stare bene, \'e una \emph{fortuna} impagabile, che io ho il privilegio di possedere. Non sei una semplice comparsa in questa mia vita, ne sei la protagonista, e sono fiero e felice di poterti avere al mio fianco, ora e per sempre.

Ringrazio le persone che hanno contribuito a realizzare questo lavoro, in particolare \emph{Alberto} e \emph{Marco}, le due figure che stanno attivamente contribuendo ad arricchire il mio percorso accademico e soprattutto professionale. Fonti di ispirazione inesauribili e su cui so di poter sempre contare e con le quali spero in futuro di continuare a collaborare.

Ringrazio tutto lo staff del laboratorio \emph{Ro.Co.Co.} dell'universit\'a \emph{La Sapienza di Roma}, dai dottorandi ai professori che mi hanno seguito e formato in questi anni di studio.

Ringrazio i ragazzi del mio team, \emph{SPQR@Work}, \emph{Wilson, Luca, Francesco, Giulio, Joseph} e tutti i ragazzi che frequentano quotidianamente il nostro laboratorio. Le giornate di studio, di lavoro e di gioco (perchè per noi questa \'e prima di tutto una passione ed un gioco) non sarebbero le stesse.

Ringrazio i miei amici, pi\'u o meno vicini, siete molti e vorrei nominarvi tutti, ma sapete che leggendo queste due righe mi sto riferendo a voi.

Le persone da ringraziare sarebbero infinite, e non basterebbe un libro intero per descrivere le nostre avventure e i motivi per cui ho l'obbligo di ringraziarvi adesso, ma ahim\'e, lo spazio sta per finire. Sappiate solamente che potrete sempre contare su di me, come io so di poter fare con voi.

Grazie, e speriamo di ritrovarci di nuovo insieme per una nuova avventura!

\emph{Daniele}
\end{acknowledgements}