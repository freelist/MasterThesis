\chapter{Benchmarks and Metrics in Industry}\label{ch:benchmarks_and_metrics}
In this chapter all the tools and the benchmarks for object localization and detection will be exposed. In particular, first the metrics and the techniques used to evaluate pose estimations will be presented, then some publicly available datasets will be presented. 

The problems that are going to be tackled are related to the 6D object pose estimation task. A 6D object pose is mathematically the 3D position of the object in the space plus 3 more terms that describe its rotation:

\begin{equation}
    \label{eq:6D_obj_pose_0}
    \hat{P} = (x, y, z, \alpha_x, \alpha_y, \alpha_z)^T
\end{equation}

The Eq. \ref{eq:6D_obj_pose_0} refers to the 6D pose estimate of an object, the first three terms $x, y, z$ describe the position in the camera reference frame, while the last three terms $\alpha_x, \alpha_y, \alpha_z$ represent the object's rotation angles along the $x, y, z$ camera axis respectively. This formulation can be easily manipulated in terms of rotation matrices and translation vectors:

\begin{equation}
    \label{eq:6D_obj_pose_1}
    \hat{P} = (R, t)
\end{equation}

The Eq. \ref{eq:6D_obj_pose_1} is a manipulated version of \ref{eq:6D_obj_pose_0} where the rotation matrix $R$ encodes the three angles of rotation $\alpha_x, \alpha_y, \alpha_z$ and the translation vector $t$ encodes the $x, y, z$ position of the object in the reference frame of the camera.

\begin{figure}
	\centering
	\begin{subfigure}{.5\textwidth}
  		\centering
  		\includegraphics[width=.9\linewidth]{figures/2_benchmarks_and_metrics/pinhole_geometry_3D}
  		\caption{3D view}
  		\label{fig:pinhole_geometry3D}
	\end{subfigure}%
	\begin{subfigure}{.5\textwidth}
  		\centering
  		\includegraphics[width=.9\linewidth]{figures/2_benchmarks_and_metrics/pinhole_geometry_2D}
  		\caption{2D view}
  		\label{fig:pinhole_geometry2D}
	\end{subfigure}
	\caption{\textbf{Pinhole Camera Geometry.} In (A) the geometry of the pinhole camera in 3D view, in (B) its 2D representation.}
	\label{fig:pinhole_geometry}
\end{figure}

Each object pose estimate can also be projected back to the camera image plane. This projection between the 3D space and the 2D space of the image plane is achieved by applying the projection obtained with a given camera matrix model. The camera model that we consider here is the so called Pinhole Camera Model, and it's geometry is depicted in Figure \ref{fig:pinhole_geometry}.

The mapping from the coordinates of a 3D point $P$ to the 2D image coordinates of the point's projection onto the image plane, according to the pinhole camera model is given by:

\begin{equation}
    \label{eq:3D_to_2D_mapping}
    x_i = \left[ \begin{array}{c} u \\ v \end{array} \right] = \dfrac{f}{x_3} \left[ \begin{array}{c} x_1 \\ x_2 \end{array} \right]
\end{equation}

In Eq. \ref{eq:3D_to_2D_mapping} $x_1, x_2, x_3$ represent the 3D position in space of a generic point, $f$ is the focal length of the camera, and $u, v$ are the image coordinates obtained after the projection.

The formulation just explained is for an ideal pinhole camera located in the origin and with focal length equal for the $x$ and $y$ axes of the image. Typically, the situation is different, and the more generic formulation is like the following:

\begin{equation}
    \label{eq:generic_pinhole_camera_model}
    x_i = \left[ \begin{array}{c} u \\ v \\ 1 \end{array} \right] = \begin{bmatrix} f_x & 0 & c_x \\ 0 & f_y & c_y \\ 0 & 0 & 1 \end{bmatrix} \left[ \begin{array}{c} x_1 \\ x_2 \\ x_3 \end{array} \right]
\end{equation}

In Eq. \ref{eq:generic_pinhole_camera_model} the camera is formulated using its internal parameter $f_x, f_y, c_x, c_y$ that represent respectively the two focal lengths on the $x, y$ image axis and the camera optical center $(c_x, c_y)^T$.

With Eq. \ref{eq:3D_to_2D_mapping} and \ref{eq:generic_pinhole_camera_model} we can project each point of the 3D object model onto the image plane and compare the estimated pose with the ground truth one following one of the metrics that are going to be explained in the sections below.

\section{Metrics}\label{sec:metrics}
The problem of evaluating how good is a pose estimate w.r.t. the ground truth is a challenging and still very open topic in computer vision community.  Taking inspiration from \cite{hodan20166DPoseEstimation} we will first focus on bounding box estimates comparison and then will pass to the more complex and challenging problem of compare two object 6D pose estimate.

\subsection{2D Intersection over Union (IoU)}\label{subsec:iou}
To do ...

\subsection{5cm, 5deg criteria}\label{subsec:5cm5deg}
To do ...

\subsection{Average Distance (AD)}\label{subsec:average_distance}
To do ...

\section{Industrially oriented Datasets}\label{sec:datasets}
To do ...

\subsection{T-LESS Dataset}\label{subsec:tless_dataset}
T-LESS Dataset \cite{hodan2017tless}.

\subsection{MVTec ITODD}\label{subsec:mvtex_itodd}
To do ...