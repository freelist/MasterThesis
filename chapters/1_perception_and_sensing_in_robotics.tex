\chapter{Perception and Sensing in Robotics: Sensors and Algorithms}\label{ch:perceptionandsensing}
To do ...

\section{3D Reconstruction}\label{sec:3dreconstruction}
To do ...

\section{Object Detection}\label{sec:objectdetection}
To do ...

\section{Robotics in Industry}\label{sec:roboticsinindustry}
To do ...

\subsection{Random Bin Picking (RBP)}\label{subsec:binpicking}
To do ...

\subsection{Pick\&Place Tasks (PPT)}\label{subsec:pickandplace}
To do ...

\section{State of the art Software Libraries in Industry}\label{sec:industrylibraries}
Machine Vision is one of the most active area in industrial settings. Over the past years, many software companies and Open Source communities have dedicated lot of effort in developing robust and effective techniques and algorithms in order to assist industrial realities, such as companies and start ups, in performing computer vision assisted tasks, e.g. random bin picking, Pick\&Place tasks and so on.

In the following subsection a list of tools and libraries will be introduced, focusing mainly on the MVTec's Halcon Libraries, which are the one that we used in the experiment phase of this work. 

\subsection{Halcon Libraries}\label{subsec:halconlibs}
Halcon\footnote{http://www.mvtec.com/products/halcon/}, from MVTec, is a set of commercial software developed and sold explicitly for industrial settings. Over the past 5 years it has become the state of the art in machine vision for industrial tasks. It serves all industries with an extensive library of more than 1600 operators for blob analysis, morphology, matching, measuring, identification, and 3D vision, to name just a few.

The full library can be accessed from common programming languages like C, C++, C\#, Visual Basic .NET, and Delphi. In particular, our tests have been developed using the C++ APIs. In the following chapters we will test this standard Machine Vision approaches over the RAW and T-Less datasets, and compare them with completely different approaches such as Deep Learning CNNs for object localization and recognition.

\begin{figure}
    \centering
    \includegraphics[width=0.8\textwidth]{figures/1_perception_and_sensing_in_robotics/hdevelop_gui_example}
    \caption{\textbf{Hdevelop GUI example.} An example of using the Hdevelop software from the Halcon Libraries. In particular here we are performing an object detection and localization task.} 
    \label{fig:hdevelop_example}
\end{figure}

The Halcon Library has also an interactive and friendly GUI, provided in order to facilitate the interfacing with the low level software APIs. The aforementioned software tool is called HDevelop, and an example of its usage and graphical interface is depicted in Figure \ref{fig:hdevelop_example}.

As anticipated, this software library is under commercial license, and our distribution has been sold to La Sapienza University of Rome that can use it for research and other non-commercial purposes. 

\begin{figure}
    \centering
    \includegraphics[width=0.8\textwidth]{figures/1_perception_and_sensing_in_robotics/mil_gui_example}
    \caption{\textbf{MIL GUI example.} The graphical user interface of the Matrox Imaging Library.} 
    \label{fig:mil_example}
\end{figure}

\subsection{Matrox Imaging Library (MIL)}\label{subsec:mil}
Another important software library that needs to be mentioned is the Matrox Imaging Library\footnote{https://www.matrox.com/imaging/en/products/software/mil/} (MIL). MIL is a complete collection of software tools for developing machine vision in lots of different scenarios, it is not restricted to the industrial one such as for the previously mentioned Halcon Libraries, but it covers also medical images applications and many others.

MIL includes also a graphic user interface for fast developing and prototyping of solutions. An example of this GUI is depicted in Figure \ref{fig:mil_example}.

This library is not part of the tests and examples performed during this work.