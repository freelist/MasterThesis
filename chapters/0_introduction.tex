\chapter{Introduction}\label{ch:intro}
To do ...

\section{Scenarios}\label{sec:scenarios}
To do ...

\subsection{RoboCup@Work}\label{subsec:robocupatwork}
To do ...

\section{Motivations and Contributions}\label{sec:motivations}
To do ...

\section{Related Work}\label{sec:relatedwork}
To do ...

\section{Structure of the Thesis}\label{sec:thesisstructure}
In the following chapters an overview of the state-of-the-art about perception in industry will be given (Chapter \ref{ch:perceptionandsensing}), focusing the attention on sensors technology (\secref{sec:sensor_techs}), 3D Scene Reconstruction and Object Detection techniques (Sec. \ref{sec:3dreconstruction} and \ref{sec:objectdetection}), and state-of-the-art software libraries commonly used in industrial settings (Sec. \ref{sec:industrylibraries}). In Chapter \ref{ch:benchmarks_and_metrics} an overview of commonly used metrics (Sec. \ref{sec:metrics}) and benchmark datasets (Sec. \ref{sec:datasets}) are going to be presented. After this detailed view of techniques, theories and tools about perception in industry, in Chapter \ref{ch:raw_dataset} RAW Dataset is presented in detail, with a large overview of its details and composition (Sec. \ref{sec:raw_features} and Sec. \ref{sec:raw_scenes_details}), the acquisition setup and procedure (Sec. \ref{sec:raw_setup_and_sensors} and \ref{sec:raw_acquisition_procedure}) and finally how ground truth has been computed (Sec. \ref{sec:ground_truth_estim}) giving also a large view of the developed tools (Sec. \ref{subsec:raw_labeltool} and \ref{subsec:pose_propagation}). In Chapter \ref{ch:experiments} some experiments on the aforementioned RAW Dataset are going to be presented and brief discussion will be derived (Sec. \ref{sec:exp_discussion}). Finally, in Chapter \ref{ch:conclusions} final conclusions will be given, and on going and future works will be announced. This document contains also an appendix (Appendix \ref{apx:appendix}) where some more details about the FlexSight project (\ref{apx:flexsight}) and the RoboCup@Work (\ref{apx:robocupatwork}) are given.